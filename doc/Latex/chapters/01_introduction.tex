\chapter{Introduction}\label{chapter:introduction}

This thesis introduces and benchmarks several Supersingular Isogeny Diffie-Hellman (\gls{SIDH}) implementations. Cryptography based on supersingular isogenies is a candidate to replace state-of-the-art asymmetric encryption primitives in the upcoming age of quantum-computers. Especially \gls{SIDH} might replace modern Diffie-Hellman key exchange algorithms in future.\\
The goal of this thesis is the comparison between existing \gls{SIDH} implementations. For a complete analysis of the current state of \gls{SIDH}, this upcoming technology is also compared to widely used Elliptic Curve Diffie-Hellman (\gls{ECDH}) protocols.
In order to generate comparable benchmarks, a benchmarking suite is developed for this thesis. The \gls{SIDH} implementations \textit{\gls{SIKE}}, \textit{\gls{PQCrypto-SIDH}}, \textit{\gls{CIRCL}} and additionally the \gls{openssl} implementation of ECDH are integrated into this benchmarking suite to obtain comparable and reliable benchmarking results.
\\\\
After this initial chapter, \hyperref[chapter:background]{Chapter~\ref*{chapter:background}} introduces the necessary technical background. Besides the introduction of basic encryption and key exchange protocols, the chapter illustrates the influence of upcoming quantum computing to modern cryptography. Finally, the term Supersingular Isogeny Diffie-Hellman (\gls{SIDH}) is introduced in detail.\\
\hyperref[chapter:existing_sidh]{Chapter 3} presents currently available libraries implementing \gls{SIDH}: \textit{\gls{SIKE}}, \textit{\gls{CIRCL}} and \textit{SIKE for Java}. Their implemented primitives, optimized versions, parameter sets and APIs are summarized and compared. Moreover the relation between the similar libraries \gls{SIKE} and \gls{PQCrypto-SIDH} is investigated. \\
In \hyperref[chapter:benchmarking_suite]{Chapter 4} the benchmarking suite -- developed in the scope of this thesis -- is described. Besides the usage and the internal operations of the software, the chapter also reveals and justifies the applied benchmarking methodologies.\\
\hyperref[chapter:analysis]{Chapter~\ref*{chapter:analysis}} provides the results measured by the benchmarking suite. The \gls{SIDH} libraries \textit{\gls{SIKE}} and \textit{\gls{CIRCL}} are are compared based on their available implementations. Furthermore, the chapter takes \gls{ECDH} in consideration to compare \gls{SIDH} with currently deployed technologies.\\
The final \hyperref[chapter:conclusion]{Chapter 6} summarizes the major results and describes limitations of this thesis.