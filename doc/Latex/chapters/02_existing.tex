\chapter{Description of existing \gls{SIDH} implementations}\label{chapter:existing_sidh}
Currently, three implementations of Supersingular Isogeny Diffie-Hellman (\gls{SIDH}) are available: \textit{\gls{SIKE}}, \textit{\gls{CIRCL}} and \textit{\gls{PQCrypto-SIDH}}. In this chapter each implementation is introduced in detail. At the end of this chapter, \autoref{tab:existing_sidh} summarizes similarities and differences between all approaches.
\\
In the following, some algorithms are described as \textit{compressed}. These compressed version exploit shorter public key sizes while increasing the computation time of the algorithms.

\section{\gls{SIKE}}
\gls{SIKE} stands for \textbf{S}upersingular \textbf{I}sogeny \textbf{K}ey \textbf{E}ncapsulation. It is the reference implementation of the first proposed isogeny-based cryptographic primitives \parencite{jao2011towards}. Today, \gls{SIKE} is a \gls{NIST} candidate for quantum-resist \textit{"Public-key Encryption and Key-establishment Algorithms"}. It is developed by a cooperation of researchers, lead by David Jao \parencite{sike2020spec}.
\\
\gls{SIKE} implements its key encapsulation mechanism (\gls{KEM}) upon a public key encryption system (\gls{PKE}) which is built upon \gls{SIDH} (as highlighted in \autoref{chapter:background}). Beside a generic reference implementation, \gls{SIKE} offers various optimized implementations of their cryptographic primitves:
\begin{itemize}
  \item Generic optimized implementation, written in portable C
  \item x64 optimized implementation, partly written in x64 assembly
  \item x64 optimized compressed implementation, partly written in x64 assembly
  \item ARM64 optimized implementation, partly written in ARMv8 assembly
  \item ARM Cortex M4 optimized implementation, partly written in ARM thumb assembly
  \item VHDL implementation
\end{itemize}
All of these implementations can be run with the following parameter sets: \texttt{p434}, \texttt{p503}, \texttt{p610} and \texttt{p751}. \gls{SIKE} asserts to countermeasure timing and cache attacks by implementing constant time cryptography~\parencite{sike2020spec}.

\subsubsection{\gls{SIKE} API}
The API of \gls{SIKE} for a \gls{SIDH} key exchange is the following:


\begin{lstlisting}[language=C, commentstyle=\color{Comment}\ttfamily, ,numbers=none, morekeywords={PrivateKey_A, PrivateKey_B, PublicKey_A, PublicKey_B, SharedSecret_A, SharedSecret_B}]
// Generate random private key for Alice
random_mod_order_A(PrivateKey_A);

// Generate random private key for Bob
random_mod_order_B(PrivateKey_B);

// Generate ephemeral public key for Alice
EphemeralKeyGeneration_A(PrivateKey_A, PublicKey_A);

// Generate ephemeral public key for Bob
EphemeralKeyGeneration_B(PrivateKey_B, PublicKey_B);

// Computation of shared secret by Alice
EphemeralSecretAgreement_A(PrivateKey_A, PublicKey_B, SharedSecret_A)

// Computation of shared secret by Bob
EphemeralSecretAgreement_B(PrivateKey_B, PublicKey_A, SharedSecret_B)

\end{lstlisting}
All parameters used in this API are of type \texttt{unsigned char*}. Note that for all implementations and all parameter sets the API is the same. Therefore, during compilation one need to include the correct files to initialize \gls{SIKE} with a specific parameter set.

\section{\gls{PQCrypto-SIDH}}
\gls{PQCrypto-SIDH} is a software library, mainly written in C. It is developed by Microsoft for experimental purposes \parencite{microsoft2020sidh}. Note that many developers of \gls{SIKE} also work for Microsoft, leading to great similarities between \gls{SIKE} and \gls{PQCrypto-SIDH}. However, in terms of compression, \gls{SIKE} references the below described Microsoft library in its documentation.\\
The \gls{PQCrypto-SIDH} library implements a isogney-based \gls{KEM} and the underlying \gls{SIDH}. Moreover, the library offers the following optimized versions:
\begin{itemize}
  \item Generic optimized implementation, written in portable C
  \item Generic optimized compressed implementation, written in portable C
  \item x64 optimized implementation, partly written in assembly
  \item x64 optimized compressed implementation, partly written in assembly
  \item ARMv8 optimized implementation, partly written in assembly
  \item ARMv8 optimized compressed implementation, partly written in assembly
\end{itemize}
All of these implementations can be run with the following parameter sets: \texttt{p434}, \texttt{p503}, \texttt{p610} and \texttt{p751}. The developers argue that the algorithms are protected against timing and cache attacks. Therefore, the library implements constant time operations on secret key material~\parencite{microsoft2020sidh}.
\subsubsection{\gls{PQCrypto-SIDH} API}
The API of \gls{PQCrypto-SIDH} for a \gls{SIDH} key exchange is the following:

\begin{lstlisting}[language=C, commentstyle=\color{Comment}\ttfamily, ,numbers=none, morekeywords={PrivateKey_A, PrivateKey_B, PublicKey_A, PublicKey_B, SharedSecret_A, SharedSecret_B}]
// Generate random private key for Alice
random_mod_order_A_SIDHpXXX(PrivateKey_A);

// Generate random private key for Bob
random_mod_order_B_SIDHpXXX(PrivateKey_B);

// Generate ephemeral public key for Alice
EphemeralKeyGeneration_A_SIDHpXXX(PrivateKey_A, PublicKey_A);

// Generate ephemeral public key for Bob
EphemeralKeyGeneration_B_SIDHpXXXPrivateKey_B, PublicKey_B);

// Computation of shared secret by Alice
EphemeralSecretAgreement_A_SIDHpXXX(PrivateKey_A, PublicKey_B, SharedSecret_A)

// Computation of shared secret by Bob
EphemeralSecretAgreement_B_SIDHpXXX(PrivateKey_B, PublicKey_A, SharedSecret_B)

\end{lstlisting}
For XXX $\in$ \{\texttt{434, 503, 610, 751}\}. All parameters used in this API are of type \texttt{unsigned char*}.

\section{\gls{CIRCL}}

\gls{CIRCL} (\textbf{C}loudflare \textbf{I}nteroperable, \textbf{R}eusable \textbf{C}ryptographic \textbf{L}ibrary) is a collection of cryptographic primitives developed by Cloudflare ~\parencite{circl2020github}. \gls{CIRCL} is written in Go and implements some quantum-secure algorithms like \gls{SIDH} and an isogeny-based \gls{KEM}. Cloudflare does not guarantee for any security within their library. Furthermore, the isogeny-based cryptographic primitives are adopted from the official \gls{SIKE} implementation. The following implementation optimizations are stated to be available:

\begin{itemize}
  \item Generic optimized implementation, written in Go (unfortunately, this version could not be compiled)
  \item AMD64 optimized implementation, partly written in assembly
  \item ARM64 optimized implementation, partly written in assembly
\end{itemize}
Note that there are no compressed versions available. The library supports the following parameter sets: \texttt{p434}, \texttt{p503} and \texttt{p751}. To avoid side-channel attacks, their code is implemented in constant time~\parencite{circl2019intro}.

\subsubsection{\gls{CIRCL} API}
The API of \gls{CIRCL} for a \gls{SIDH} key exchange is the following:

\begin{lstlisting}[language=GO, commentstyle=\color{Comment}\ttfamily, ,numbers=none, morekeywords={PrivateKey_A, PrivateKey_B, PublicKey_A, PublicKey_B, SharedSecret_A, SharedSecret_B}]
// Generate random private key for Alice
PrivateKey_A = sidh.NewPrivateKey(sidh.FpXXX, sidh.KeyVariantSidhA)
PrivateKey_A.Generate(rand.Reader)

// Generate random private key for Bob
PrivateKey_B = sidh.NewPrivateKey(sidh.FpXXX, sidh.KeyVariantSidhB)
PrivateKey_B.Generate(rand.Reader)

// Generate public key for Alice
PublicKey_A = sidh.NewPublicKey(sidh.FpXXX, sidh.KeyVariantSidhA)
PrivateKey_A.GeneratePublicKey(PublicKey_A)

// Generate public key for Bob
PublicKey_B = sidh.NewPublicKey(sidh.FpXXX, sidh.KeyVariantSidhB)
PrivateKey_B.GeneratePublicKey(PublicKey_B)

// Computation of shared secret by Alice
SharedSecret_A := make([]byte, PrivateKey_A.SharedSecretSize())
PrivateKey_A.DeriveSecret(SharedSecret_A, PublicKey_B)

// Computation of shared secret by Bob
SharedSecret_B := make([]byte, PrivateKey_B.SharedSecretSize())
PrivateKey_B.DeriveSecret(SharedSecret_B, PublicKey_A)

\end{lstlisting}
For XXX $\in$ \{\texttt{434, 503, 751}\}.

\section{Overview}

\begin{table}[H]
  \centering
  \begin{tabular}{|K{3cm}|K{4cm}|K{4cm}|K{4cm}|}
	\hline
    \rowcolor{lightgray!50}
     & \textbf{\gls{SIKE}} & \textbf{\gls{PQCrypto-SIDH}} & \textbf{\gls{CIRCL}} \\
	\hline
    \bfseries\makecell{Developer}& Research cooperation & Microsoft & Cloudflare \\
    \hline
    \bfseries\makecell{Language} & \makecell{C \\ Assembly} & \makecell{C \\ Assembly} & \makecell{GO \\ Assembly}\\
    \hline
    \bfseries\makecell{Reference} & \makecell{\url{www.sike.org}} & \makecell{Github:\\\href{https://github.com/microsoft/PQCrypto-SIDH}{PQCrypto-SIDH} }& \makecell{Github:\\\href{https://github.com/cloudflare/circl}{cloudflare/circl}}\\
    \hline
    \hline
    \bfseries\makecell{Implemented \\ primitives} & \makecell{\gls{SIDH} \\ \gls{PKE} \\ \gls{KEM}\\} & \makecell{\gls{SIDH} \\ \gls{KEM}\\} & \makecell{\gls{SIDH} \\ \gls{KEM}\\} \\
	\hline
	\bfseries\makecell{Available \\ parameters} & \makecell{p434 \\ p503 \\ p610 \\ p751\\} & \makecell{p434 \\ p503 \\ p610 \\ p751\\} & \makecell{p434 \\ p503 \\ p751\\} \\
	\hline
	\bfseries\makecell{Optimized \\ versions} & \makecell{Generic \\ Generic compressed \\ x64 \\ x64 compressed \\ ARM64 \\ ARM Cortex M4 \\ VHDL \\} & \makecell{Generic \\ Generic compressed  \\ x64 \\ x64 compressed \\ ARMv8 \\ ARMv8 compressed\\} & \makecell{Generic \\ AMD64\\ ARM64 \\} \\
	\hline
	\bfseries\makecell{Security} & \makecell{Constant time} & \makecell{Constant time}  & \makecell{Constant time} \\
	\hline

  \end{tabular}
   \caption[Existing \gls{SIDH} implementations]{Overview of existing \gls{SIDH} implementations.}\label{tab:existing_sidh}
\end{table}