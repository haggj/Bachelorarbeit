\chapter{Description of existing SIDH implementations}\label{chapter:existing_sidh}
Currently, three implementations of Supersingular Isogeny Diffie-Hellman (SIDH) ble, namely \textit{SIKE}, \textit{CIRCL} and \textit{PQCrypto-SIDH}. In this chapter each implementation is introduced in detail. At the end of the chapter, \autoref{tab:existing_sidh} shows similarities and differences between all approaches.
\\
In the following, some algorithms are described as \textit{compressed}. These compressed version exploit shorter public key sizes while increasing the computation time of the algorithms.

\section{SIKE}
SIKE stands for \textbf{S}upersingular \textbf{I}sogeny \textbf{K}ey \textbf{E}ncapsulation. It is the reference implementation of the first proposed isogeny-based cryptographic primitives \parencite{jao2011towards}. Today, SIKE is a NIST candidate for quantum-resist \textit{"Public-key Encryption and Key-establishment Algorithms"}. It is developed by a cooperation of researchers, lead by David Jao \parencite{sike2020spec}.
\\
SIKE implements its key encapsulation mechanism (KEM) upon a public key encryption system (PKE), which is built upon SIDH (as highlighted in \autoref{chapter:background}). Beside a generic reference implementation, SIKE offers various optimized implementations of their cryptographic primitves:
\begin{itemize}
  \item Generic optimized implementation, written in portable C
  \item x64 optimized implementation, partly written in x64 assembly
  \item x64 optimized compressed implementation, partly written in x64 assembly
  \item ARM64 optimized implementation, partly written in ARMv8 assembly
  \item ARM Cortex M4 optimized implementation, partly written in ARM thumb assembly
  \item VHDL implementation
\end{itemize}
All of these implementations can be run with the following parameter sets: \texttt{p434}, \texttt{p503}, \texttt{p610} and \texttt{p751}. SIKE states to countermeasure timing and cache attacks by implementing constant time cryptography~\parencite{sike2020spec}.


\section{PQCrypto-SIDH}
PQCrypto-SIDH is a software library mainly written in C. It is developed by Microsoft for experimental purposes \parencite{microsoft2020sidh}. Note, that many developers of SIKE also work for Microsoft leading to great similarities between SIKE and PQCrypto-SIDH. However, in terms of compression, SIKE references  the here described Microsoft library in its documentation.\\
The PQCrypto-SIDH library implements a isogney-based KEM and the underlying SIDH. Moreover, the library offers the following optimized versions:
\begin{itemize}
  \item Generic optimized implementation, written in portable C
  \item Generic optimized compressed implementation, written in portable C
  \item x64 optimized implementation, partly written in assembly
  \item x64 optimized compressed implementation, partly written in assembly
  \item ARMv8 optimized implementation, partly written in assembly
  \item ARMv8 optimized compressed implementation, partly written in assembly
\end{itemize}
All of these implementations can be run with the following parameter sets: \texttt{p434}, \texttt{p503}, \texttt{p610} and \texttt{p751}. The developers argue to protect the algorithms against timing and cache attacks. Therefore, the library implements constant time operations on secret key material~\parencite{microsoft2020sidh}.

\section{CIRCL}

CIRCL (\textbf{C}loudflare \textbf{I}nteroperable, \textbf{R}eusable \textbf{C}ryptographic \textbf{L}ibrary) is a by Cloudflare developed collection of cryptographic primitives~\parencite{circl2020github}. CIRCL is written in Go and implements some quantum-secure algorithms like SIDH and an isogeny-based KEM. Cloudflare does not guarantee for any security within their library. Furthermore, the isogeny-based cryptographic primitives are adopted from the official SIKE implementation. The following implementation optimizations are available:

\begin{itemize}
  \item Generic optimized implementation, written in Go
  \item AMD64 optimized implementation, partly written in assembly
  \item ARM64 optimized implementation, partly written in assembly
\end{itemize}

Note, that there are no compressed versions available. The library supports the following parameter sets: \texttt{p434}, \texttt{p503} and \texttt{p751}. To avoid side-channel attacks, their code is implemented in constant time~\parencite{circl2019intro}.

\begin{table}[htpb]
  \centering
  \begin{tabular}{|K{3cm}|K{4cm}|K{4cm}|K{4cm}|}
	\hline
    \rowcolor{lightgray!50}
     & \textbf{SIKE} & \textbf{PQCrypto-SIDH} & \textbf{CIRCL} \\
	\hline
    \bfseries\makecell{Developer}& Research cooperation & Microsoft & Cloudflare \\
    \hline
    \bfseries\makecell{Language} & \makecell{C \\ Assembly} & \makecell{C \\ Assembly} & \makecell{GO \\ Assembly}\\
    \hline
    \bfseries\makecell{Reference} & \makecell{C \\ Assembly} & \makecell{C \\ Assembly} & \makecell{GO \\ Assembly}\\
    \hline
    \hline
    \bfseries\makecell{Implemented \\ primitives} & \makecell{SIDH \\ PKE \\ KEM\\} & \makecell{SIDH \\ KEM\\} & \makecell{SIDH \\ KEM\\} \\
	\hline
	\bfseries\makecell{Available \\ parameters} & \makecell{p434 \\ p503 \\ p610 \\ p751\\} & \makecell{p434 \\ p503 \\ p610 \\ p751\\} & \makecell{p434 \\ p503 \\ p751\\} \\
	\hline
	\bfseries\makecell{Optimized \\ versions} & \makecell{Generic (portable c) \\ x64 \\ x64 compressed \\ ARM64 \\ ARM Cortex M4 \\ VHDL \\} & \makecell{Generic (portable c) \\ x64 \\ x64 compressed \\ ARMv8 \\ ARMv8 compressed\\} & \makecell{Generic (GO) \\ AMD64\\ ARM64 \\} \\
	\hline
	\bfseries\makecell{Security} & \makecell{Constant time} & \makecell{Constant time}  & \makecell{Constant time} \\
	\hline

  \end{tabular}
   \caption[Existing SIDH implementations]{Overview and comparison of existing SIDH implementations.}\label{tab:existing_sidh}
\end{table}